\section{conclusion}

\label{section:conclusion}



During the course of the last two labs, we created a system that produces a PWM signal in order to match a RMS voltage value in a circuit. The value is acquired through user input on a keypad. Based on a calculation of the time constant of our circuit, we were able to pick appropriate values for the PWM and ADC frequencies, which enabled our PWM controller to efficiently adjust the PWM duty cycle and match the target voltage. This system was designed to be as modular as possible, by isolating each logical section in a designated thread. Keypad, ADC and Display threads were created, which were responsible for managing user input, filtering and processing data, and driving the 7-segment display. The ADC thread implementation was later reverted back to an Interrupt-Service-Routine as a direct consequence of strange behaviour. Finally, this system is able to match a target voltage with 3 digits of precision in generally under 1.5s, which we deem to be acceptable in the context of this experiment. Despite limiting CPU usage in most cases, our system could be improved. Some additional improvements would include reducing system power consumption by removing the polling aspect of the keypad thread in favour of a combination of interrupts and a polling Thread.


%
%
%The challenges we faced from the interrupt driven approach compared to the multithreaded design had to do with attempting to make the ADC interrupt a thread and running into the ADC thread blocking all other threads. This required us to make a design decision and keep our ADC buffer full callback as an interrupt. Following this hurdle we were able to make our system fully functional and successfully integrated Lab 3's keyboard and display's interrupt-driven implementations to a multithreaded one with the help of the FreeRTOS kernel to manage the two threads in our system.

