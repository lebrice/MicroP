%notfulltexdoc
\section*{Work Breakdown}
\addcontentsline{toc}{section}{Work Breakdown}
In general, most students contributed to a variety of different tasks over the course of the
project. A brief summary of the tasks that each student contributed to is given in the subsections
below.

\subsection*{Harley Wiltzer}
With regard to the design of the system Harley's contributions mainly pertained to the design of the
Discovery Board software and the Flask Web Server. Harley originally set up the accelerometer on the
Discovery Board and designed the tap detection system. He also designed the filters used on the
accelerometer data. Furthermore, he contributed to the
implementation of UART communication between the Discovery and Nucleo Boards.\\\\
In terms of the web development portion, Harley designed the API endpoint used for the accelerometer
data, including a method of parsing the data from the \texttt{POST} request, storing the data as a
CSV file on the server and generating a graph of the data. Moreover, this included the design of the
web page used as the user interface.\\\\
Furthermore, Harley contributed to resolving a bug that prevented BLE communication from the Nucleo
to the Android Application. This was a massive bug that set the team back many days, and it was
resolved through extensive debugging and testing.\\\\
Finally, Harley completed the implementation of the WiFi communication between the Android
Application and the web server, after older implementations had been made obsolete due to major
changes in the system's design.

\subsection*{Matthew Lesko-Krleza}
Matthew's major contributions were towards the smart phone android application's implementation, testing and integration. 
He implemented the core features of the application which involved: scanning for BLE peripheral devices, enabling
a GATT connection between client and server devices, the initial code for reading/writing data from/to the Nucleo
board, the initial code for saving data to a file on the smartphone, WiFi communication with the web server, as well as
the user interface for the application. Since the smart phone is an intermediary component, this required him to test and 
debug, and modify the android application such that it integrates properly with the Nucleo board and the web server.\\
Moreover, he wrote code for custom BLE services and characteristics on the Nucleo board, but were later removed due to 
system design changes.\\
Overall he had to learn about the BLE stack, smartphone WiFi communication, and how the Nucleo board sends data over BLE.