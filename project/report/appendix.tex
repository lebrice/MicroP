%notfulltexdoc
\section*{Appendices}
\addcontentsline{toc}{section}{Appendices}
\subsection*{Appendix A - Setting up the Web Server}
\addcontentsline{toc}{subsection}{Setting up the Web Server}
Some minor setup is necessary in order to run the web server, however the process is drastically
simplified when taking advantage of some serendipitous Python tools. Firstly, Python 3 will be
required. Furthermore, it is recommended to use
\href{https://pypi.org/project/virtualenv/}{\texttt{virtualenv}} to keep dependencies managed.
Finally, this tutorial will be assuming the user has access to the free open-source Python package
manager, \href{https://pypi.org/project/pip}{\texttt{pip}}.\\\\
Firstly, navigate to the \texttt{api/} directory in this project's root. It is strongly recommended
to create a \texttt{virtualenv} here, but it is not necessary. Next, execute the following command:
\begin{lstlisting}[language=bash,basicstyle=\ttfamily]
$ pip install -r requirements.txt
\end{lstlisting}
This will install all Python package dependencies required for the project. With these dependencies
installed, the server is ready to run! Execute the following command, still within the \texttt{api/}
directory:
\begin{lstlisting}[language=bash,basicstyle=\ttfamily]
$ python api.py
\end{lstlisting}
The command above launches the web server on port 5000. The server will run indefinitely. When it is
desired to kill the server, navigate to the shell through which the server is being run and press
\texttt{CTRL + C}.\\\\
Note that for the end-to-end pipeline
described in this paper, the Android application must be configured to send HTTP packets to the IP
of the machine running this server. In order to do this, navigate to the file at:\\
\texttt{<project root>/android/app/src/main/java/com/ecse426/project/app/AppController.java}\\\\
In this file, edit the string \texttt{WEBSITE\_ADDRESS} to contain the IP address of the machine
running the server. For the changes to take effect, the Android app must be rebuilt and reinstalled
on an Android device.

\subsection*{Appendix B - Setting up the Android Application}
\addcontentsline{toc}{subsection}{Setting up the Android Application}
Some setup is required to run the Android application since it is not available on the Google Play Store.\\
The user has to download and install the latest version of \href{https://developer.android.com/studio/index.html}{Android Studio}.
Once the IDE has been installed, the user must clone the source code from this \href{https://github.com/lebrice/MicroP}{repository}.\\ 
Once the source code has been downloaded and the IDE installed, the user must \texttt{Import} the Android project by starting Android Studio and selecting the \texttt{Open existing Android project} within the Project Menu. 
From there, Android Studio prompts you with a File Explorer. Within the File Explorer, navigate to the source code that has just been dowloaded and under MicroP/project/android, select the \texttt{build.gradle} file. Android Studio will import the project and automatically start downloading any dependencies required for the project.\\
Once that is complete, connect your Android phone via USB to your computer. Now within Android Studio, click on either \texttt{Run > Run} to install the application on your phone or on \texttt{Run > Debug} to install and launch a debug session for the application, allowing the user to view any logs within the console.\\
With either choice, the Android application is now installed and ready to use!