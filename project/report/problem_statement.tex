%notfulltexdoc
\section{Problem Statement}
This text presents the design of a system whose ultimate goal is to gather data using the
peripherals of a microcontroller and publish this data to a webserver via a chain of different
communication media. From the user's perspective, the only setup required involves installing an
Android application (designed for the purposes of this project) and launching a web server to
collect the recorded data. After the Android device has paired with the Nucleo microcontroller, the
data recorded from the accelerometer or microphone attached to the STM32F407 Discovery board will be
transmitted automatically to the webserver, where the data may be visualized numerically and with
pictorial support. If the data being sent is a microphone recording, the web server will perform
\textit{speech recognition} to parse a number $n$ from the recorded audio, which will be communicated
all the way back to the Discovery board, causing an LED to blink $n$ times.\\\\
Given the reduced amount of resources usually present in microcontrollers, several devices were
required to work in tandem in order to create a pipline from the Discovery board to the Internet.
The Discovery board, while having access to an accelerometer and an ADC for reading microphone data,
has no resources for communicating over Bluetooth or HTTP. Instead, it may communicated via a wired
connection to the Nucleo board that is fitted with a Bluetooth Low Energy (BLE) transmitter. Given
that the Nucleo has no WiFi capabilities, it communicates data over BLE to an Android application,
which may then tunnel the data over HTTP to the web server.
Thus, in order to communicate sensor data from the Discovery board all the way to the web server, three
distinct communication protocols are used (UART, BLE, HTTP), each having their own unique quirks and
\textit{impedimentum}. A major challenge of the design of this system revolved around the design of
how to encapsulate data to be sent, and how to ensure the timing between transmitters and receivers
was appropriate.\\\\
Furthermore, given the plethora of devices responsible for the transmission of data in this system,
it was necessary to implement four distinct software components across three different technologies.
As will be described further in this report, both the Discovery and Nucleo board required
independent embedded-C programs. Furthermore, to bridge the gap between the microcontrollers and the
web application, an Android app was devised and implemented in Java. Finally, the web application
prompted the design of a RESTful API as well as a fancy interface for displaying the data it
received. This was achieved mainly with the Python language and some of its excellent libraries.\\\\
Finally, another design challenge was that of speech recognition. Although a custom Convolutional
Neural Network was initially designed for this task, the amount of time that would have been
required to train it properly would have severely hindered progress on the rest of the project.
Therefore, the speech recognition feature was outsourced to a Google Speech API.\\\\
The remainder of this text will describe how the system was designed, and will discuss the various
challenges were experienced as well as how they were resolved. Ultimately, the system proposed in
this paper is meant to show how embedded systems can communicate with each other, as well as how
they may communicate over the vast Internet. Thus, the proposed system exhibits the
\textit{Internet of Things}, a field and concept that has been growing tremendously and is projected
to grow into a market worth \$7.1 trillion by the year 2020\cite{iot}.\\\\
The legendary philosopher John Milton once said ``give me the liberty to know, to utter, and to
argue freely according to conscience, above all liberties''\cite{ethics}. The Internet of Things and
the design proposed in this paper ultimately recognize microcontrollers as entities capable of
knowing, uttering, and arguing, which was, above all else, demonstrated by the proposed system. The
Internet of Things revolution, therefore, shall eventually bring the ``Inter-computational Covenant
on Civil and Political Rights'', allowing all devices to communicate equally to advance society. The
authors hope that the results demonstrated in this text show promise in the evolution and
advancements in the freedom of device-expression.
