%notfulltexdoc
\section{Conclusion}
To conclude, the authors wish to point out that their quest for providing the Discovery Board with
the opportunity to communicate data over the vast Internet was successful. Despite having no ability
to communicate wirelessly at all, given the design of the other components of the system, such as
the Nucleo Board Middleman and the Android Application Prophet, it was possible to create a pipeline
from the Discovery's UART pins all the way to the Flask Web Server's HTTP endpoints.\\\\
This result has a considerable impact on wireless technology and the state of microcontrollers. Due
to the relatively low cost of microcontrollers such as the STM32F407 Discovery and the STM32F401RE
Nucleo, creating application-specific systems for wireless communication is both affordable and
feasible. Furthermore, when designed properly, the Discovery and Nucleo boards should be
particularly energy-efficient, which is yet another attractive feature. Moreover, the small form
factor of these boards allows them to be particularly portable and mobile.\\\\
The authors wish to pose the lessons learned in the design of this system as an allegory to the tale
of the four sons in the \textit{Haggadah}. The four sons are described as follows:
\begin{enumerate}
	\item The wise son
	\item The wicked son
	\item The simple son
	\item The son who does not know how to ask a question
\end{enumerate}
Part of the fascination of the story stems from the idea that these sons are characterized this way.
Is everything so black and white? Is the wise son smart, or is he just a smart aleck? In relation to
this system, it is not necessarily easy to determine which component is ``the wise son". It may seem
as though it's the web server, as it has likely the most powerful computational resources. But does
being given resources make one wise? One may consider the wicked son to be the Nucleo Board, due to
the sheer frustration caused by configuring BLE and UART simultaneously on it. But by causing
frustration, does that make one wicked? Let's not forget the good that the Nucleo brought, that
being the ability to bridge the gap between the Discovery and the Android Application. The Android
Application may seem like the simple son, due to its ease of use relative to its immense power,
though is it not possible that it's ease of use is a byproduct of very challenging development
behind the scenes? And finally, one may assume the Discovery Board to correspond to the son that
does not know how to ask a question. However, there is a very important difference between
\textit{not knowing} how to ask a question, and being \textit{unwilling} to ask a question.\\\\
As in the case of the famous story from the \textit{Haggadah}, it becomes difficult to characterize
the four components in such a way. The ultimate conclusion from the story is that, in fact, this
task is impossible. Each son can be wise, each son can be wicked, each son can be simple, and each
son may refrain from asking questions -- these characteristics are not mutually exclusive. And this,
the authors believe, is the most profound concept that was demonstrated over the course of this
project. The Discovery Board, with so few resources, is literally incapable of communicating over
WiFi. However, it was shown that this \textit{does not prohibit it from doing so}. Ninety-nine
people out of a hundred likely would have classified the Discovery Board as ``the son who could not
ask a question", and yet in reality, with a little bit of clever design, this notion was clearly
refuted.
